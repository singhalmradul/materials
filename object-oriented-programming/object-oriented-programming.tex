\documentclass{article}
\usepackage{geometry}[a4paper]
\usepackage{graphicx}
\usepackage{listings}
\usepackage{titlesec}
\usepackage[hidelinks]{hyperref}
\usepackage{makeidx}
\usepackage{color}
\usepackage{mathdots}
\usepackage{sans}
\usepackage{minted}
\usepackage{xcolor}

\setminted{
  fontsize=\small,
  frame=lines,
  linenos=true,
  tabsize=2
}
% \linespread{1.1}

\makeindex

% Centered and bold section headings
\titleformat{\section}[block]{\normalfont\Large\bfseries\filcenter}{}{0em}{}
\titleformat{\subsection}[block]{\normalfont\large\bfseries}{}{0em}{}
\titleformat{\subsubsection}[block]{\normalfont\normalsize\bfseries}{}{0em}{}

\raggedright
\title{Object Oriented Programming}
\author{Mradul Singhal}
% \date{\today}

\begin{document}

\maketitle
\newpage

\tableofcontents
\newpage

\section{Introduction}
Object-Oriented Programming (OOP) organizes software design around data, or objects, rather than functions and logic. Here are the fundamental concepts of OOP:
\begin{itemize}
    \item \textbf{Classes:} Templates for creating objects, defining their structure and behavior.
    \item \textbf{Objects:} Instances of classes that represent specific elements with attributes and behaviors.
    \item \textbf{Encapsulation:} Hiding the internal state of an object and requiring all interaction to be performed through an object's methods.
    \item \textbf{Abstraction:} Exposing only the necessary and relevant parts of an object to the outside world.
    \item \textbf{Inheritance:} A mechanism for one class to inherit the properties and behavior of another class.
    \item \textbf{Polymorphism:} The ability to present the same interface for differing underlying data types.
\end{itemize}
\newpage

\section{Association, Aggregation \& Composition}
\subsection{Association}
\begin{itemize}
    \item When an object have a relationship with another object
    \item When an object uses a reference of another object, whether it may be \textit{Aggregation} or \textit{Composition}
    \item \textbf{Example:}
          \begin{minted}{scala}
class Foo:
  var bar: Bar = uninitialized
          \end{minted}
          Foo uses Bar
    \item It may be \textit{one-to-one, one-to-many, many-to-one} or \textit{many-to-many} relationship
\end{itemize}
\subsection{Aggregation}
\begin{itemize}
    \item  A relationship where the child can exist independently of the parent.
    \item When an object is injected into another object
    \item \textbf{Example:}
          \begin{minted}{scala}
class Foo(bar: Bar)
          \end{minted}
          When \lstinline|Foo| dies, \lstinline|Bar| may live on
          \item It may be \textit{one-to-one, one-to-many, many-to-one} or \textit{many-to-many} relationship
\end{itemize}
\subsection{Composition}
\begin{itemize}
    \item An object owns another object and is responsible for that object's lifetime.
    \item When an object is instantiated within another object
    \item \textbf{Example:}
          \begin{minted}{scala}
class Foo:
  val bar = Bar()
          \end{minted}
          When \lstinline|Foo| dies, so does \lstinline|Bar|
          \item It may be \textit{one-to-one} or \textit{one-to-many} relationship
\end{itemize}
\newpage
\printindex
\end{document}